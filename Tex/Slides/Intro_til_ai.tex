% !TEX spellcheck = en-US
\documentclass[aspectratio=169]{beamer}
\usepackage{tikz}

\usetheme{bi}

\title{Introduksjon til AI for finansnæringen}
\author{Vegard H. Larsen}
\institute{Førsteammanuensis, Institutt for datavitenskap og analyse, BI}
\date{\today}

\begin{document}

\frame{\titlepage}

\begin{frame}
    \frametitle{Velkommen}
\end{frame}

\begin{frame}{Definisjon av Kunstig Intelligens}
    \begin{itemize}
        \item \textbf{Kunstig Intelligens (AI)} er et felt innen datavitenskap som fokuserer på å utvikle maskiner og algoritmer som kan utføre oppgaver som vanligvis krever menneskelig intelligens.
        \item Dette inkluderer oppgaver som problemløsning, talegjenkjenning, visuell persepsjon, beslutningstaking og språkforståelse.
    \end{itemize}

\end{frame}

\begin{frame}
    \frametitle{Opprinnelsen til Kunstig Intelligens}

    \begin{itemize}
        \item \textbf{1940-tallet: Tidlige ideer}
              \begin{itemize}
                  \item Alan Turing utvikler Turing-testen, et kriterium for intelligens i en maskin.
              \end{itemize}

        \item \textbf{1956: Fødselen av AI}
              \begin{itemize}
                  \item John McCarthy mynter termen "artificial intelligence" ved Dartmouth-konferansen, der de første ideene om maskinlæring og problemløsning blir utforsket.
              \end{itemize}
    \end{itemize}

\end{frame}

\begin{frame}
    \frametitle{Utviklingen gjennom 1960- og 70-tallet}

    \begin{itemize}
        \item \textbf{1960-tallet: Utvidelse av forskning}
              \begin{itemize}
                  \item Utvikling av de første programmeringsspråkene for AI, som LISP av John McCarthy.
                  \item Fokus på problemløsningsmetoder og teorien bak AI.
              \end{itemize}

        \item \textbf{1970-tallet: AI Vinteren begynner}
              \begin{itemize}
                  \item Skeptisisme til AI vokser, finansiering tørker inn.
                  \item Fokus på mer spesifikke applikasjoner, som logiske resonneringssystemer.
              \end{itemize}
    \end{itemize}

\end{frame}

\begin{frame}
    \frametitle{Moderne AI: Gjenfødsel og gjennombrudd}

    \begin{itemize}
        \item \textbf{1980-tallet til nå: En ny æra}
              \begin{itemize}
                  \item Gjenoppliving av interessen for nevrale nettverk og "deep learning".
                  \item Store fremskritt med introduksjonen av algoritmer som backpropagation.
                  \item AI blir en integrert del av hverdagen gjennom smart teknologi og internett.
              \end{itemize}
    \end{itemize}

\end{frame}

\end{document}