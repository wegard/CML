\documentclass[12pt, a4paper]{article}
%\usepackage{geometry}
\usepackage[inner=1.0cm,outer=1.0cm,top=2.5cm,bottom=2.5cm]{geometry}
\pagestyle{empty}
\usepackage{graphicx}
\usepackage{fancyhdr, lastpage, bbding, pmboxdraw, tabularx}
\usepackage[usenames,dvipsnames]{color}
\definecolor{darkblue}{rgb}{0,0,.6}
\definecolor{darkred}{rgb}{.7,0,0}
\definecolor{darkgreen}{rgb}{0,.6,0}
\definecolor{red}{rgb}{.98,0,0}
\usepackage[colorlinks,pagebackref,pdfusetitle,urlcolor=darkblue,citecolor=darkblue,linkcolor=darkred,bookmarksnumbered,plainpages=false]{hyperref}
\renewcommand{\thefootnote}{\fnsymbol{footnote}}
\newcommand{\RNum}[1]{\uppercase\expandafter{\romannumeral #1\relax}}
\pagestyle{fancyplain}
\fancyhf{}
\lhead{ \fancyplain{}{\textsc{EDI 3400: Programming and Data Management} }}
\chead{ \fancyplain{}{} }
\rhead{ \fancyplain{}{\textsc{Course description} }}
%\rfoot{\fancyplain{}{page \thepage\ of \pageref{LastPage}}}
\fancyfoot[RO, LE] {page \thepage\ of \pageref{LastPage} }
\thispagestyle{plain}

%%%%%%%%%%%% LISTING %%%
\usepackage{listings}
\usepackage{caption}
\DeclareCaptionFont{white}{\color{white}}
\DeclareCaptionFormat{listing}{\colorbox{gray}{\parbox{\textwidth}{#1#2#3}}}
\captionsetup[lstlisting]{format=listing,labelfont=white,textfont=white}
\usepackage{verbatim} % used to display code
\usepackage{fancyvrb}
\usepackage{acronym}
\usepackage{amsthm}
\VerbatimFootnotes % Required, otherwise verbatim does not work in footnotes!
\definecolor{OliveGreen}{cmyk}{0.64,0,0.95,0.40}
\definecolor{CadetBlue}{cmyk}{0.62,0.57,0.23,0}
\definecolor{lightlightgray}{gray}{0.93}
\lstset{
%language=bash,                          % Code langugage
basicstyle=\ttfamily,                   % Code font, Examples: \footnotesize, \ttfamily
keywordstyle=\color{OliveGreen},        % Keywords font ('*' = uppercase)
commentstyle=\color{gray},              % Comments font
numbers=left,                           % Line nums position
numberstyle=\tiny,                      % Line-numbers fonts
stepnumber=1,                           % Step between two line-numbers
numbersep=5pt,                          % How far are line-numbers from code
backgroundcolor=\color{lightlightgray}, % Choose background color
frame=none,                             % A frame around the code
tabsize=2,                              % Default tab size
captionpos=t,                           % Caption-position = bottom
breaklines=true,                        % Automatic line breaking?
breakatwhitespace=false,                % Automatic breaks only at whitespace?
showspaces=false,                       % Dont make spaces visible
showtabs=false,                         % Dont make tabls visible
columns=flexible,                       % Column format
morekeywords={__global__, __device__},  % CUDA specific keywords
}

%%%%%%%%%%%%%%%%%%%%%%%%%%%%%%%%%%%%
\begin{document}
\begin{center}
    {\LARGE \textsc{Programming and Data Management}}
\end{center}
\begin{center}
    Semester: Fall 2024\\
    %Program: Bachelor of Digital Business\\
    Course code: EDI 3400 \\
    BI Norwegian Business School\\
    {\small This version: \today}
\end{center}
%\date{September 26, 2014}

\begin{center}
    \rule{\textwidth}{0.4pt}
    \begin{minipage}[t]{\textwidth}
        \medskip
        \begin{tabularx}{\linewidth}{lXlX}
            \textbf{Instructor:}   & Vegard H. Larsen                                                        & \textbf{Contact:}      & \href{mailto:vegard.h.larsen@bi.no}{vegard.h.larsen@bi.no} \medskip \\
            \textbf{Office:}       & B3Y-075                                                                 & \textbf{Office Hours:} & Wednesdays 8--10 \medskip                                           \\
            \textbf{Course Pages:} & \href{https://www.vegardlarsen.com/edi3400/}{vegardlarsen.com/edi3400/} & \textbf{TA:}           & TBA \medskip                                                        \\
        \end{tabularx}\medskip
    \end{minipage}
    \rule{\textwidth}{0.4pt}
\end{center}
\vspace{.2cm}
\setlength{\unitlength}{1in}
\renewcommand{\arraystretch}{2}

\noindent
This course introduces Python and SQL, two of the most popular and indispensable programming languages for data analysts.
In addition, the course covers the basics of data management with focus on relational databases.

\vskip.25in
\noindent\textbf{\large Compulsory Readings}
\begin{enumerate}
    \item Python for everybody: exploring data using Python 3 by \it{Charles R. Severance}
          (\href{https://www.py4e.com/book.php}{Link})
    \item Introduction to Python for Econometrics, Statistics and Data Analysis by \it{Kevin Sheppard}
          (\href{https://www.kevinsheppard.com/files/teaching/python/notes/python_introduction_2021.pdf}{Link})
    \item SQL queries for mere mortals: a hands-on guide to data manipulation in SQL by \it{John L. Viescas}
\end{enumerate}

\vskip.25in
\noindent\textbf{\large Software} \\
In this course we will mainly be using Python.
We will install Python through the Anaconda distribution.
You can run into some problems if you don't have the correct version of the software installed.
Try to install the latest version you find.
\begin{enumerate}
    \item \href{https://www.anaconda.com/}{Anaconda}
    \item \href{https://sqlitebrowser.org/}{DB Browser for SQLite}
          %\item MySQL community server, \url{https://dev.mysql.com/downloads/}
          %\item MySQL Workbench, \url{https://dev.mysql.com/downloads/}
\end{enumerate}

\vskip.15in
\noindent\textbf{\large Supplementary Material} \\%\footnotemark
Here are some interesting resources that will be useful during the course.
This is not mandatory readings.
\begin{itemize}
    \item \href{https://docs.python.org/3/tutorial/}{Python tutorial}
    \item \href{https://jupyter-notebook.readthedocs.io/}{Jupyter Notebook documentation}
    \item \href{https://numpy.org/doc/}{NumPy documentation}
    \item \href{https://pandas.pydata.org/docs/}{Pandas documentation}
    \item \href{https://matplotlib.org/stable/users/index}{Matplotlib documentation}
    \item \href{https://docs.python.org/3/library/sqlite3.html}{SQLite Python API documentation}
\end{itemize}

\newpage
\noindent\textbf{\large Lecture Plan}\\
Lectures are from ?? -- ?? on ?? unless otherwise stated.
The plan below is tentative and subject to change.
\begin{enumerate}
    \item[] \underline{Lecture 1:} [Date, TBA] {\bf Introduction}
        \begin{itemize}
            \item Ch.1 of {\it Severance} and Ch.1 of {\it Sheppard}
        \end{itemize}
    \item[] \underline{Lecture 2:} [Date, TBA] {\bf Variables, expressions, and statements}
        \begin{itemize}
            \item Ch.2 and Ch.3 of {\it Severance} and Ch.2, Ch.4 and Ch.10 of {\it Sheppard}
        \end{itemize}
    \item[] \underline{Lecture 3:} [Date, TBA] {\bf Built in functions and containers}
        \begin{itemize}
            \item Ch.4, Ch.6, Ch.8, Ch.9, Ch.10 of {\it Severance} and Ch.23 of {\it Sheppard}
        \end{itemize}
    \item[] \underline{Lecture 4:} [Date, TBA] {\bf Loops and flow control, input and output}
        \begin{itemize}
            \item Ch.3, Ch.5, Ch.7, Ch.11 of {\it Severance} and Ch. 12 of {\it Sheppard}
        \end{itemize}
    \item[] \underline{Lecture 5:} [Date, TBA] {\bf The Standard Library, Functions and Classes}
        \begin{itemize}
            \item Ch.4, Ch.11, Ch.14 of {\it Severance} and Ch.29 of {\it Sheppard}
        \end{itemize}
    \item[] \underline{Lecture 6:} [Date, TBA] {\bf 3rd party libraries (linear algebra): Numpy}
        \begin{itemize}
            \item Ch.3, Ch.5, Ch.6, Ch.7, Ch.11 and Ch.19 of {\it Sheppard}
        \end{itemize}
    \item[] \underline{Lecture 7:} [Date, TBA] {\bf 3rd party libraries (data frames): Pandas}
        \begin{itemize}
            \item Ch.8 and Ch.16 of {\it Sheppard}
        \end{itemize}
    \item[] \underline{Lecture 8:} [Date, TBA] {\bf 3rd party libraries (graphics): Matplotlib}
        \begin{itemize}
            \item Ch.15 of {\it Sheppard}
        \end{itemize}
    \item[] \underline{Lecture 9:} [Date, TBA] {\bf Advanced topics: IDEs and generative AI for programming}
        \begin{itemize}
            \item Ch.14, Ch.15 and Ch.16 of {\it Sheppard}
        \end{itemize}
    \item[] \underline{Lecture 10:} [Date, TBA] {\bf Advanced topics: Analyzing data, probability and statistics}
        \begin{itemize}
            \item Ch.19 and Ch.20 of {\it Sheppard}
        \end{itemize}
    \item[] \underline{Lecture 11:} [Date, TBA] {\bf Introduction to relational databases}
        \begin{itemize}
            \item Ch.15 of {\it Severance} and Ch.1, Ch.2, Ch.3 of {\it Viescas}
        \end{itemize}
    \item[] \underline{Lecture 12:} [Date, TBA] {\bf SQL Basics}
        \begin{itemize}
            \item Ch.4, Ch.5 and Ch.6 of {\it Viescas}
        \end{itemize}
    \item[] \underline{Lecture 13:} [Date, TBA] {\bf Python and SQL}
        \begin{itemize}
            \item Ch.15 of {\it Severance}
        \end{itemize}
    \item[] \underline{Lecture 14:} [Date, TBA] {\bf Summing up and Q\&A}
\end{enumerate}

\newpage
\noindent\textbf{\large Asynchronous learning: Homework/Projects} \\
TBA

%\vskip.25in
%\noindent\textbf{\large Exercise Sessions} \\
%TBA

\vskip.25in
\noindent\textbf{\large Exam}
\begin{itemize}
    \item Format: 30 hour take-home exam
    \item Start: ??.??.2024 at 09:00
    \item Submission deadline: ??.??.2023 at 15:00
\end{itemize}

\end{document}